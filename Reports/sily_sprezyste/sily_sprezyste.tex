% filepath: c:\Users\kuba-\Desktop\simple_graph_tool_xdf\src\eeg\reportPendulum.tex
\documentclass[12pt]{article}
\usepackage[T1]{fontenc} % Use T1 font encoding
\usepackage[utf8]{inputenc} % Ensure UTF-8 encoding
\usepackage[polish]{babel} % Enable Polish language support
\usepackage{amsmath}
\usepackage{graphicx}
\usepackage{booktabs}
\usepackage{float}
\usepackage[margin=2.5cm]{geometry}
\usepackage{siunitx}
\usepackage{titlesec}
\titlespacing*{\subsection}{0pt}{*0.5}{*0.5} % Adjusts spacing before and after subsections
\usepackage{caption}
\usepackage{lmodern}
\usepackage{placeins} % For FloatBarrier
\usepackage{hyperref} % For hyperlinks in the document

\title{Sprawozdanie z Laboratorium Fizyki: Wahadło fizyczne}
\date{}

\begin{document}

% --------------------------- STRONA TYTUŁOWA --------------------------
\begin{titlepage}
    \centering
    \Large
    \textbf{DYDAKTYCZNE LABORATORIUM FIZYKI} \\
    \vspace{0.2cm}
    \textbf{UNIWERSYTET RADOMSKI}\\
    im. Kazimierza Pułaskiego w Radomiu \\
    
    \vspace{1.5cm}
    \begin{flushleft}
        \textbf{Wydział:} {WTEiI} \\
        \textbf{Kierunek:} Informatyka \\
        \textbf{Rok Akademicki:} 2024/2025 \\
        \textbf{Semestr:} II \\
        \textbf{Grupa:} 3 \\
        \textbf{Zespół:} 2 \\
        \textbf{Data:} 11.03.2025 \\
        \textbf{Prowadzący ćwiczenie:} dr B. Winiarska \\
    \end{flushleft}
    
    \vspace{1cm}
    \begin{flushleft}
        \textbf{Nr ćwiczenia:} 2 \\
        \textbf{Temat ćwiczenia:} \\
        \textbf{Siły Sprężyste (Współczynik Sprężystości)} \\
    \end{flushleft}
    
    \vspace{1cm}
    \begin{flushleft}
        \textbf{Wykonujący ćwiczenie:}
        \begin{itemize}
            \item Jakub Oleszczuk
            \item Mikołaj Majewski
            \item Mateusz Ofiara
        \end{itemize}
    \end{flushleft}

    \vfill
    \begin{flushleft}
        \textbf{Oceny:} \\
        1.\hspace{2cm}2.\hspace{2cm}3.
    \end{flushleft}
\end{titlepage}

% --------------------------- TREŚĆ SPRAWOZDANIA --------------------------
\section*{Wstęp}

\section*{Wyniki pomiarów}

\begin{itemize}
    \item Masa: \( m = 0.783 \, \text{kg} \)
    \item Długość pręta: \( l = 0.9 \, \text{m} \pm -0.1 \, \text{m} \)
    \item Środek ciężkości: \( s = 0.45 \, \text{m} \pm -0.1 \, \text{m} \)
    \item Wychylenie: \( \alpha = 15^\circ \pm 1^\circ \)
\end{itemize}

Tabela pomiarów Metodą B:

\begin{center}
    \begin{tabular}{|c|c|c|c|c|c|c|}
    \hline
    L.p. & d [m] & t1 [s] & t2 [s] & T [s] & X [m²] & Y [m$\cdot$s$^{-2}$] \\
    \hline
    1 & 0{,}35 & 28{,}91 & 28{,}90 & 1{,}45 & 0{,}010 & 0{,}21 \\
    2 & 0{,}32 & 28{,}58 & 29{,}11 & 1{,}44 & 0{,}017 & 0{,}27 \\
    3 & 0{,}29 & 28{,}79 & 28{,}80 & 1{,}44 & 0{,}026 & 0{,}33 \\
    4 & 0{,}26 & 29{,}21 & 28{,}93 & 1{,}45 & 0{,}036 & 0{,}40 \\
    5 & 0{,}23 & 29{,}26 & 29{,}39 & 1{,}47 & 0{,}048 & 0{,}47 \\
    6 & 0{,}20 & 30{,}71 & 30{,}44 & 1{,}53 & 0{,}063 & 0{,}58 \\
    7 & 0{,}17 & 31{,}24 & 31{,}47 & 1{,}57 & 0{,}078 & 0{,}69 \\
    8 & 0{,}14 & 34{,}24 & 34{,}04 & 1{,}71 & 0{,}096 & 0{,}90 \\
    9 & 0{,}11 & 38{,}17 & 38{,}19 & 1{,}91 & 0{,}116 & 1{,}24 \\
    10 & 0{,}08 & 48{,}26 & 48{,}38 & 2{,}42 & 0{,}137 & 2{,}16 \\
    \hline
    \end{tabular}
    \end{center}
    
\section*{Wykres}
\section*{Obliczenia}
\section*{Wnioski}
\end{document}
